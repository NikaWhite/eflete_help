\documentclass[titlepage,oneside,11pt]{book}
\author{Nikita Belyavskij}
% Need to add subtitle
\title{EFL Edje Theme Editor}

\usepackage{xcolor,graphicx}
%\usepackage[pdftex]{hyperref}
\begin{document}
% generates the title
\maketitle
% insert the table of contents
\tableofcontents
\chapter{Introduction}
%This section should contain general description of EFL them5ing.
\section{Enlightement foundation libraries}
\subsection{Overview}
%short overview of EFL libraries, accordingly to 
%the Edje + Elementary
\subsection{EDC language and Edje library}
%describe flexible way for theming, by using EDC
%language and Edje library mechanism
\subsection{Elementary Theming}
%describe opportunites, that Elementary Theme engine
%give to developers and designers.
\newpage
\section{EFL Edje Theme Editor}
%Eflete is super clever tool. Make user trust to this quote.
\chapter{Installing}
\section{Linux}
\subsection{Dependencies}
\subsection{Upstream git repository}
\subsection{Tarballs}
\subsection{Packages}
\subsubsection{.deb}
\subsubsection{.rpm}
\newpage
\section{MacOS}
\newpage
\section{Windows}
%This section describe installing EFLETE in differrent ways
\chapter{Using the EFL Edje Theme Editor}
\section{Create new project}
\section{Import project}
\subsection{EDC project}
\subsection{EDJ project}
\section{Resource management}
\subsection{Images}
\subsection{Sounds}
\subsection{Color classes}
\subsection{Text styles}
\section{Edit project}
\subsection{Styles}
\subsection{Layouts}
\subsection{Parts}
\subsubsection{States}
\subsubsection{Attributes}
\subsection{Programs}
\subsection{Workspace}
\section{Live preview}
\section{History}
\section{Enventor mode}
\chapter{User interface}
\section{Main screen}
\includegraphics[scale=0.2]{images/main_screen.png}
Main screen contain component, that always accesseble to user. Here is: main menu, tollbar with fast access buttons, status bar.
\subsection{Main menu}
\includegraphics{images/main_menu.png}
\subsubsection{File}
File menu provide ability to manage project. 
\includegraphics{images/file_menu.png}
Export as edc submenu
\includegraphics{images/file_export_as_edc_submenu.png}
User can export whole project to edc files tree or export only currently opened group into one single edc file and resource directories.
\subsubsection{View}
View section is provide ability to manipulate apperance of workspace area.
\includegraphics{images/view_menu.png}
Functionality of changing zoom fsctor or switching beetwen views on the workspace is placed in workspace submenu.
\includegraphics{images/view_workspace_submenu.png}
Manipulating rulers functions placed in rulers submenu
\includegraphics{images/view_rulers_submenu.png}
\subsection{Toolbar}
%describe toolbar buttons
\subsection{Statusbar}
\subsection{Lists}
\subsubsection{Widgets}
\subsubsection{Layouts}
\subsubsection{Parts}
\subsubsection{Items}
\subsubsection{State}
\subsubsection{Signal}
\subsection{Attributes}
\subsubsection{Layout}
\subsubsection{Part}
\subsubsection{State}
\subsubsection{Object area}
\subsubsection{Fill}
\subsubsection{Image}
\subsubsection{Table}
\subsubsection{Box}
\subsubsection{Text}
\subsubsection{Textblock}
\subsection{Enventor mode}
\section{Wizzards}
\subsection{New project}
\subsection{Import edc}
\subsection{Import edj}
\section{Resource managers}
\subsection{Images}
\subsection{Sounds}
\subsection{Color classes}
\subsection{Textblock styles}
\section{Animator}
\subsection{Program editor}
\subsection{Animator}
\section{Workspace}
\subsection{View modes}
\subsection{Rulers}
\subsection{Context menu}
\subsubsection{Zoom}
\subsubsection{Rulers}
\subsection{Edit mode}
\subsubsection{Resize}
\subsubsection{Aligment}
\section{History}
\chapter{Appendix}
\section{Hotkeys}
\section{Example application}
%Describe little thing like hotkeys, tips and examples
\end{document}